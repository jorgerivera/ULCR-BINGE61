\documentclass[xcolor=dvipsnames]{beamer}

\usepackage[utf8]{inputenc}
\usepackage{default}
\usetheme[width=50pt]{ULatina}   % Bergen, Darmstadt
\usecolortheme[named=Green]{structure}
\usepackage{graphicx}
\usepackage{pgfpages}
\usepackage{tikz}
\usepackage[spanish]{babel}

%\setbeameroption{show notes on second screen}
%\setbeamersize{sidebar width left=50pt}


\title[$\mu C$]{BINGE-61 Microcontroladores}
\subtitle{Programa del curso, evaluación y cronograma}
\author{Prof. Jorge Rivera~Guti\'errez}
\institute{Universidad Latina de Costa Rica\\ Ingenier\'\i a en Electr\'onica}
\logo{\includegraphics[height=45pt]{world.png}}
\date{III cuatrimestre 2015}
\newcommand{\pageframe}[1]{\frame{\begin{center}{ \Huge #1 }\end{center}}}

\begin{document}

\begin{frame}
 \maketitle
\end{frame}

\begin{frame}
 \begin{center}
  \Large Créditos: 3\\~\\
  Requisito: Arquitectura de Computadores\\~\\
  Período: VIII\\~\\
  Modalidad: cuatrimestral
 \end{center}
\end{frame}

\section{Descripción del curso}

\begin{frame}{Descripción del curso}
 Este curso  que se desarrollar\'a en un alto porcentaje en el laboratorio de electr\'onica, pretende que el estudiante adquiera los conocimientos necesarios para el desarrollo de sistemas basados en microcontroladores; adem\'as utilizar\'a los conocimientos adquiridos en desarrollar un proyecto electr\'onico al finalizar el curso.
\end{frame}

\section{Objetivos}

\pageframe{Objetivos}

\subsection{Objetivo general}

\begin{frame}{Objetivo general}
  \begin{block}{Objetivo general}
    \begin{itemize}
      \item El estudiante deberá conocer las características generales y específicas de los dispositivos de microcontroladores para solución de problemas electrónicos.
      \item El estudiante deberá profundizar en el estudio de técnicas de medición de parámetros físicos tales como temperatura, presión, humedad, etc.
      \item El estudiante deberá aprender las técnicas básicas de control digital para casos sencillos.
    \end{itemize}
  \end{block}
\end{frame}

\subsection{Objetivos específicos}

\begin{frame}{Objetivos específicos}
  \begin{block}{Objetivos específicos}
    \begin{itemize}
      \item<1> Estudiar y describir las caracter\'\i sticas b\'asicas de los microcontroladores.
      \item<1> Estudiar el sistema de administraci\'on de memoria de los microcontroladores.
      \item<1> Estudiar el sistema de administraci\'on de puertos de los microcontroladores.
      \item<1> Comprender el manejo de dispositivos perif\'ericos de utilidad pr\'actica como puerto serie, DMA.
    \end{itemize}
  \end{block}
\end{frame}
\begin{frame}{Objetivos específicos}
  \begin{block}{Objetivos específicos}
    \begin{itemize}
      \item Dise\~nar dispositivos de adquisici\'on de datos provenientes de variables f\'\i sicas.
      \item Explorar sistemas est\'andar de conversi\'on y transmisi\'on de informaci\'on.
      \item Programar microcontroladores mediante la planificaci\'on y ejecuci\'on de un proyecto aplicado.
    \end{itemize}
  \end{block}
\end{frame}

\section{Contenidos}

\pageframe{Contenidos}

\subsection[Características]{Características básicas de los microcontroladores}

\begin{frame}{Características básicas de los microcontroladores}
  \only<1>{
  \begin{block}{}
  \begin{itemize}
    \item Tipos de microcontroladores
    \item Características funcionales
    \item Familia Intel, Motorola, Microchip y otras
    \item Arquitectura Interna
    \item Interfaz con dispositivos periféricos y expansión de memoria
    \item Transmisión asíncrona: Puerto serial. Interno y Externo.
    \item Puertos de uso específico (Display, Teclado)
  \end{itemize}
  \end{block}}
  \only<2>{
  \begin{block}{}
  \begin{itemize}
    \item Administración de memoria
    \item Dispositivos de almacenamiento de información
    \item Cálculo de dimensiones y direcciones de los dispositivos de almacenamiento de información
    \item Bancos de memoria
    \item Campos de aplicación de los microcontroladores
    \item Ejemplos con dispositivos microcontroladores prácticos
  \end{itemize}
  \end{block}}
\end{frame}

\subsection[Conexión de $\mu$C]{Conexión de microcontroladores con sensores y actuadores}

\begin{frame}{Conexión de $\mu$C con sensores y actuadores}
\begin{block}{}
\begin{itemize}
 \item Medición de parámetros físicos (temperatura, presión, humedad relativa, luminosidad, desplazamiento, etc)
 \item Conversión de variables físicas a eléctricas (Conversión Analógica / Digital / Analógica).
 \item Activación de actuadores electromecánicos.
\end{itemize}
\end{block}
\end{frame}

\subsection[Programación]{Programación de Microcontroladores}

\begin{frame}{Programación de Microcontroladores}
\begin{block}{}
\begin{itemize}
 \item Herramientas de desarrollo: Ensambladores, simuladores, depuradores y tarjetas de prototipos.
 \item Administración de recursos: memoria, puertos, temporizadores, convertidores, etc.
 \item Desarrollo de programas: Edición, ensamblado y depuración.
\end{itemize}
\end{block}
\end{frame}

\subsection[Proyectos]{Desarrollo de Proyectos con Microcontroladores}

\begin{frame}{Desarrollo de Proyectos con Microcontroladores}
\begin{block}{}
\begin{itemize}
 \item Planificación del proyecto: Definición del problema a solucionar, descripción, propósitos, programación de actividades. El proyecto debe incluir la medición de al menos 2 variables físicas (medición analógica), el control de 2 dispositivos actuadores (uno de forma analógica y otro de forma digital), así como 3 entradas digitales y 3 salidas digitales.
 \item Ejecución del proyecto: Diseño, pruebas de laboratorio, puesta en funcionamiento y verificación de resultados.
 \item Elaboración del informe del proyecto: Involucra aspectos teóricos de operación así como de funcionamiento del sistema implementado y resultados obtenidos.
\end{itemize}
\end{block}
\end{frame}

\section{Metodología}

\pageframe{Metodología}

\begin{frame}{Metodología}

Se impartirán clases magistrales para la explicación teórica de los temas del curso. El estudiante elaborará un proyecto de aplicación en ingeniería cuyo desarrollo será guiado por el profesor, en sus distintas etapas a lo largo del cuatrimestre. El curso contempla al menos un 60\% del tiempo de clases para la ejecución del proyecto aplicado. 

\end{frame}

\section{Estrategias de aprendizaje}

\pageframe{Estrategias de aprendizaje}

\begin{frame}{Estrategias de aprendizaje}

Los estudiantes llevarán a cabo algunas investigaciones y pruebas escritas que les ayudarán a comprender mejor los temas del curso.

\textbf{El estudiante dedicará horas extra clase} en la realización de las investigaciones y en el proyecto del curso.

\end{frame}

\section{Evaluaci\'on}

\pageframe{Evaluación}

\begin{frame}{Evaluación}
\begin{center}
\begin{tabular}{|l|r|}\hline
	Examen Parcial		&	30\%\\\hline
	Examen Final 		&	30\%\\\hline
	Proyecto (Laboratorio)  &	40\%\\\hline\hline
	TOTAL			&	100\%\\\hline
\end{tabular}
\end{center}
\end{frame}

\begin{frame}{Evaluación propuesta}
\begin{center}
\begin{tabular}{|l|l|r|}\hline
	Examen Parcial		& Examen Parcial	& 	30\%\\\hline
				& Investigaciones (2)	&	10\%\\\cline{2-3}
	Examen Final (30\%)			& Quices (3)		&	15\%\\\cline{2-3}
				& Práctica para parcial &	 5\%\\\hline
				& (I) 2 Prácticas de Lab	&	 10\%\\\cline{2-3}
				& (G) Informe preliminar	&	 5\%\\\cline{2-3}
				& (I) Exposición	&	5\%\\\cline{2-3}
	Laboratorio (40\%)	& (G) Funcionamiento	&	5\%\\\cline{2-3}
				& (I) Bitácora (GitHub)	&	 5\%\\\cline{2-3}
				& (G) Informe final		&	10\%\\\hline\hline
	\multicolumn{2}{|l|}{TOTAL}			&	100\%\\\hline
\end{tabular}
\end{center}
\scriptsize{NOTA: En la sección de laboratorio, los rubros indicados con una (I) indican evaluación individual y con una (G) evaluación grupal.}
\end{frame}
\section{Observaciones}

\pageframe{Observaciones}

\begin{frame}{Observaciones generales}
  \begin{block}{Observaciones generales}
    \begin{itemize}[<+->]
      \item Se atenderán las disposiciones del Reglamento del Régimen Estudiantil.
      \item Se atenderán las disposiciones del reglamento de uso del Laboratorio de Ingeniería Electrónica.
      \item El curso es convalidable, \textbf{no} puede presentarse por suficiencia y \textbf{no} tiene derecho a examen de ampliación.
    \end{itemize}
  \end{block}
\end{frame}

\begin{frame}{Presentación de informes y evaluaciones}
\begin{block}{Presentación de informes y evaluaciones}
  \begin{itemize}[<+->]
    \item Los informes y evaluaciones deberán ser realizados con una ortografía y redacción de nivel universitario.
    \item Los informes deberán ser preparados usando el formato IEEE Transactions. La descripción de este formato y las plantillas para el mismo pueden ser encontrados en \url{http://bit.ly/bN3uDr}. 
    \item Se deberán presentar utilizando el sistema de preparación de documentos \LaTeX. 
    \item Las imágenes y los diagramas utilizados en los informes deberán ser de buena calidad y generados utilizando programas apropiados para dicha tarea.
  \end{itemize}
\end{block}
\end{frame}


\begin{frame}{Presentación de informes y evaluaciones}
\begin{block}{Presentación de informes y evaluaciones}
  \begin{itemize}[<+->]
    \item Todo documento preparado para este curso deberá ser entregado de forma digital \textbf{en formato PDF} al correo electrónico \texttt{jm.rivera.g@gmail.com}.
    \item En los informes habrá un rubro con un valor de 10\% correspondiente al cumplimiento de los requerimientos de formato, redacción, ortografía y calidad del documento.
  \end{itemize}
\end{block}
\end{frame}

\begin{frame}{Fechas de entrega}

\begin{block}{Fechas de entrega}
  \begin{itemize}[<+->]
    \item Si por motivos justificados se requiere de más tiempo para completar una asignación, esta deberá ser solicitada al profesor al menos con 24 horas de anticipación.
    \item En caso de no haber solicitado una extensión por anticipado o de haberse vencido la extensión, la máxima nota estará dada por la fórmula:

      \[ M = 100-\frac{1.367}{10}\cdot n^{1.367} \]

    donde $M$ es la nota máxima y $n$ es la cantidad de horas de atraso en la entrega.
  \end{itemize}
\end{block}
\end{frame}

\begin{frame}{Asistencia}
\begin{block}{Asistencia}
  \begin{itemize}[<+->]
    \item El curso es de asistencia obligatoria. Se requiere de una asistencia al 80\% de las horas lectivas.
    \item El curso consta de 14 sesiones de teoría de 3 horas y 14 sesiones de laboratorio de 2 horas, para un total de 70 horas. Se requiere la asistencia a 56 horas de clase para aprobar el curso.
    \item Las ausencias podrán ser justificadas por iniciativa del estudiante cuando haya una razón que así lo amerite.
    \item Las ausencias a evaluaciones deberán ser tramitadas de acuerdo con lo especificado en el reglamento de la Universidad.
  \end{itemize}
\end{block}
\end{frame}

\section{Bibliografía}

\pageframe{Bibliografía}
\nocite{*}

\begin{frame}{Bibliografía}

\begin{thebibliography}{5}
 \bibitem{Texto}
  Torrente, Óscar
  \textit{Arduino -- Curso Práctico de Formación}.
  Alfaomega,
  Primera Edición,
  2013.
 \bibitem{Cons1}
  Vega Ferreira, Juan Carlos.
  \textit{Microcontroladores -- Programación. Familias y sus distintas aplicaciones en la industria}.
  Alfaomega,
  Primera Edición,
  2007.
 \bibitem{Cons2}
  Angulo Usategui, José María; Romero Yesa, Susana y Angulo Martínez, Ignacio.
  \textit{Microcontroladores PIC -- 2ªParte}.
  McGraw Hill,
  Segunda Edición.
  2006.
 \bibitem{Cons3}
  Valdés Pérez, Fernando y Pallás Areny, Ramón,
  \textit{Microcontroladores -- Fundamentos y Aplicaciones con PIC}.
  Alfaomega,
  Primera Edición,
  2007.
 \bibitem{Cons4}
  Mano, Morris M.
  \textit{Lógica digital y diseño de computadores}.
  Prentice Hall Hispanoamericana,
  1982.
\end{thebibliography}

\end{frame}

\section{Cronograma}

\pageframe{Cronograma}

\begin{frame}{Cronograma}
 \begin{center}
  \begin{tabular}{|c|c|c|}\hline
   Semana & \multicolumn{1}{c|}{Teoría} & \multicolumn{1}{c|}{Laboratorio} \\ \hline \hline
   1 & Introducción & Práctica de laboratorio \#1 \\ \hline
   2 & \multicolumn{2}{c|}{Feriado} \\ \hline
   3 & Exposición \#1 & Entrega de informe \#1 \\ \hline
   4 &  & Entrega de informe preliminar \\ \hline
   5 & Quiz \#1 & \multicolumn{1}{c|}{$\vdots$}     \\ \hline
   6 & Exposición \#2 & \multicolumn{1}{c|}{$\vdots$}     \\ \hline
   7 & \multicolumn{1}{c|}{$\vdots$} & \multicolumn{1}{c|}{$\vdots$}     \\ \hline
   8 & Examen parcial y tarea & \multicolumn{1}{c|}{$\vdots$}     \\ \hline
  \end{tabular}
 \end{center}
\end{frame}

\begin{frame}{Cronograma}
 \begin{center}
  \begin{tabular}{|c|c|c|}\hline
   Semana & \multicolumn{1}{c|}{Teoría (L)} & \multicolumn{1}{c|}{Laboratorio (K)} \\ \hline \hline

   9 & \multicolumn{1}{c|}{$\vdots$} & \multicolumn{1}{c|}{$\vdots$}     \\ \hline
   10 & \multicolumn{1}{c|}{$\vdots$} & \multicolumn{1}{c|}{$\vdots$}     \\ \hline
   11 & Quiz \#2 & \multicolumn{1}{c|}{$\vdots$}     \\ \hline
   12 & \multicolumn{1}{c|}{$\vdots$} & \multicolumn{1}{c|}{$\vdots$} \\ \hline
   13 & \multicolumn{1}{c|}{$\vdots$} & \multicolumn{1}{c|}{$\vdots$} \\ \hline
   14 & Quiz \#3 & \multicolumn{1}{c|}{$\vdots$}  \\\hline
   15 & \multicolumn{1}{c|}{$\vdots$} & Exposiciones y demos  \\ \hline
   \textbf{16} & \multicolumn{1}{c|}{$\vdots$} & \textbf{Bitácora e Informe final} \\ \hline
  \end{tabular}

 \end{center}

\end{frame}

\end{document}
