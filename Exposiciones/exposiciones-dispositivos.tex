\documentclass[letterpaper,10pt]{article}
\usepackage[utf8x]{inputenc}
\usepackage[margin=1cm,top=.5cm,bottom=.5cm]{geometry}

%opening
\title{BINGE-61 Microcontroladores\\ Exposiciones sobre sensores y otros}
\author{Ing. Jorge Rivera}
\date{III Cuatrimestre, 2014\\ 13 de octubre, 2014}
\markboth{}{}

\begin{document}

\maketitle

\section{Exposiciones}

\begin{center}
\begin{tabular}{|c|c|c|}\hline


\#  & Dispositivo				& Encargado	\\ \hline\hline
1   & AD5330			 		& Paola Andrade   \\[.2cm]\hline
2   & TMP006					& Francisco González\\[.2cm]\hline
3   & MRF24J40					& Chrystel Sánchez 	\\[.2cm]\hline
4   & DS1307				 	& Luis Ignacio Solano	\\[.2cm]\hline
5   & LSM303DLHC				& José David Tames \\[.2cm]\hline
6	& MAX7317					& Carlos García \\[.2cm]\hline

\end{tabular}
\end{center}

\section{Datos solicitados}
\begin{small}
\begin{itemize}
\item ¿Quién lo fabrica?
\item ¿Qué tipos de encapsulado o empaquetado ofrecen? (PDIP, SOIC, QFN...)
\item ¿Qué hace?
\item ¿Cómo se comunica con el CPU o microcontrolador?
\item ¿Hay librerías disponibles para utilizarlo? ¿Para cuáles plataformas?
\item Si es necesario, incluya algún diagrama ilustrativo.
\end{itemize}
\end{small}

\section{Otras consideraciones}
\begin{small}
\begin{itemize}
\item Cuide la ortografía y la redacción en su presentación.
\item Asegúrese de conocer los términos que se utilizan en la misma. Si no los comprende, investigue y luego consulte con el profesor.
\item Si está traduciendo un texto, lea el resultado y asegúrese de que la redacción sea correcta.
\item Puede utilizar una hoja o ficha con notas, pero no lea toda su presentación.
\item Tendrá 10 minutos para su exposición. 
\end{itemize}
\end{small}

\end{document}
