\documentclass[12pt,letterpaper]{IEEEtran}
\usepackage[utf8]{inputenc}
\usepackage[spanish]{babel}
\usepackage{enumitem}

\title{Laboratorio 1: Introducción a Arduino}
\author{Prof. Ing. Jorge Rivera G.}
\date{\today}


\newcommand\MYhyperrefoptions{bookmarks=true,bookmarksnumbered=true,
pdfpagemode={UseOutlines},plainpages=false,pdfpagelabels=true,
colorlinks=true,linkcolor={black},citecolor={black},
urlcolor={black}}

\usepackage[\MYhyperrefoptions]{hyperref}

\begin{document}
\hypersetup{pdftitle={Laboratorio 1: Introducción a Arduino},
pdfsubject={IEC-650 Laboratorio de Sistemas Digitales},
pdfauthor={Ing. Jorge Rivera},
pdfkeywords={arduino, sistemas digitales}}

\renewcommand{\leftmark}{UNIVERSIDAD LATINA DE COSTA RICA -- IEC-650 LABORATORIO DE SISTEMAS DIGITALES}

\maketitle


\begin{abstract}
Esta es una práctica básica para conocer los fundamentos del lenguaje y el ambiente Arduino. 
\end{abstract}
%\chapter{capitulo}
\section{Descripción}

Esta práctica será realizada enteramente en el laboratorio. Cada estudiante realizará un pequeño circuito utilizando una tarjeta Arduino para resolver una pequeña tarea.


\section{Introducción}

El profesor dará una breve charla sobre el uso de las funciones que serán utilizadas.

\section{Materiales}

Para esta práctica se necesitarán los siguientes materiales.

\begin{center}
\begin{tabular}{c|c}\hline
	Cant. & \hspace{2cm}Material\hspace{2cm} \\\hline\hline
	1 	& Arduino 		\\\hline
	1	& Protoboard 	\\\hline	
	---	& leds			\\\hline
	--- & interruptores		\\\hline
	--- & potenciómetros	\\\hline
	--- & motores (DC, servo) \\\hline
	---	& Resistencias variadas \\\hline
	--- & Cables		\\\hline
\end{tabular}
\end{center}

\section{Requerimientos}

Para la conclusión satisfactoria de este laboratorio se realizar uno de los proyectos en la siguiente lista:

\begin{enumerate}
	\item Control de velocidad de parpadeo
	\item Control de brillo
    \item Control de posición de servomotor
    \item Comparación de valores analógicos
    \item Timbre con dos tonos
    \item Motor de tiempo constante
    \item Lectura ADC por puerto serial
    \item Selección de mensaje con botones
    \item Selección de combinación de leds con puerto serial
    \item Selección de brillo de led con puerto serial
\end{enumerate}

\section{Procedimiento}


\subsection{Preparación}

\begin{enumerate}
	\item Se deberá realizar el diseño de un circuito, indicando los números de pines a utilizar en la tarjeta y la forma de conexión del resto de los componentes. Este diseño se deberá realizar en papel y ser mostrado al profesor antes de ser alambrado.

\end{enumerate}

\subsection{Alambrado del circuito}

\begin{enumerate}[resume]
	\item El circuito se deberá alambrar usando protoboards u otros dispositivos de prototipo apropiados. El alambrado deberá mantenerse ordenado en la medida de lo posible.
\end{enumerate}

\subsection{Programación}

\begin{enumerate}[resume]
	\item Se deberá escribir el programa en la interfaz de Arduino.
\end{enumerate}

\subsection{Verificación}

\begin{enumerate}[resume]
	\item Cuando se haya concluido con todos los requerimientos del circuito escogido, se deberá realizar una última demostración al profesor y se concluirá la práctica.
\end{enumerate}

\subsection{Demostración}

\begin{enumerate}[resume]
	\item Se deberá explicar a los compañeros la tarea realizada.
\end{enumerate}

\section{Informe}

El informe que se deberá presentar tendrá dos versiones: una versión impresa y una versión digital.

\subsection{Informe impreso}

El informe impreso constará de las siguientes partes:

\begin{enumerate}
  \item Encabezado
  \item Resumen o abstract
  \item Descripción del circuito
  \item Listado de materiales
  \item Esquemático o diagrama del circuito
  \item Descripción del programa realizado
  \item Conceptos aprendidos
\end{enumerate}

El informe deberá realizarse utilizando el sistema de preparación de documentos \LaTeX, utilizando el formato IEEEtran. El documento deberá ser entregado en forma impresa en la clase correspondiente y en forma digital en el Aula Virtual, incluyendo el código fuente y el resultado en PDF.  La fecha de entrega será una semana natural después de la realización de la práctica.

La lista de componentes deberá incluir los circuitos integrados y elementos activos utilizados en la práctica con sus números de parte detallados, la cantidad y valor de los elementos pasivos utilizados y cualquier otro elemento eléctrico utilizado en el circuito. No se deberán incluir cables, bases para circuitos integrados, protoboards, etc.

Para realizar el esquemático del circuito, se deberá utilizar una herramienta apropiada. Utilice símbolos apropiados para un esquemático. No son admisibles diagramas realizados en Microsoft Paint, Adobe Photoshop u otras herramientas similares. Se sugiere utilizar Fritzing.

Para presentar la descripción del programa, se deberá hacer una explicación de cómo funciona el programa. Se podrá ilustrar esta sección con recortes del programa. Los recortes del listado del programa se deberán presentar utilizando el ambiente \texttt{verbatim}, y con las tabulaciones correctas. 

Los conceptos aprendidos deberán ser una lista de notas importantes que se hayan recogido durante la clase y de los conceptos que se aplicaron en la práctica. Deberá realizarse una breve explicación de cada elemento en la lista.

Se castigará duramente el intento de plagio.

\subsection{Versión digital}

Se presentará una descripción del circuito y el código realizado en una documentación digital. Se usará el wiki ubicado en \url{https://github.com/jorgerivera/ULCR-BINGE61/wiki/Laboratorio-1}. 

Se deberán incluir las secciones: descripción del circuito, materiales, esquemático y descripción del programa.

\section{Evaluación}

La evaluación de este práctica será con una calificación de 0 a 100, distribuida de la siguiente forma:

\begin{center}
 \begin{tabular}{p{0.35\textwidth}|c}\hline
   Funcionamiento del circuito (de acuerdo a los requisitos indicados anteriormente) 					     & 30\% \\\hline
   Calidad del informe	  				& 10\% \\\hline
   Esquemático							& 10\% \\\hline
   Listado del programa					& 10\% \\\hline
   Conceptos aprendidos					& 15\% \\\hline
   Descripción del circuito				& 10\% \\\hline
   Componentes o materiales				& 5\% \\\hline
   Versión digital						& 10\% \\\hline\hline
   Total								& 100\% \\
 \end{tabular}
\end{center}

La sección ``calidad del informe'' corresponde a 10 puntos que podrán obtenerse en caso de que el informe esté escrito con correcta redacción y ortografía y esté presentado de forma correcta.

Este documento puede usarse como base para el informe, pero en el mismo deberán incluirse solamente las secciones especificadas anteriormente y nada más.

Si por motivos justificados se requiere de más tiempo para completar el informe, esta deberá ser solicitada al profesor al menos con 24 horas de anticipación.

En caso de no haber solicitado una extensión por anticipado o de haberse vencido la extensión, la máxima nota estará dada por la fórmula:

\[ M(n) = 100-\frac{1.367}{10}\cdot n^{1.367} \]

donde $M$ es la nota máxima y $n$ es la cantidad de horas de atraso en la entrega. La nota final será:

\[ F(n) = M(n)\cdot T \]

donde $T$ corresponde al porcentaje obtenido de los rubros especificados anteriormente.

\end{document} 